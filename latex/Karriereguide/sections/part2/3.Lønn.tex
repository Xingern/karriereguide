Lønn er et svært sensitivt tema, men jeg oppfordrer alle til å snakke mer åpent om det. Hvis ikke så lar vi visse selskaper slippe unna med å konsekvent betale sine ansatte dårligere, noe som er grunnen til at fagforeninger eksisterer. Jeg skal nå forsøke å gi litt tall på noen ulike bransjer.



\section{Sommerjobb}

Lønna man får på sommerjobb varierer svært mye, enda mer enn det fulltidsstillinger gjør. Likevel så har vi standarder å sammenlikne med og kan dermed få en viss indikasjon. Det er spesielt verdt å nevne at man kan forhandle sommerjobblønn også, men igjen så er det ofte kun 2 måneder og mange orker derfor ikke å ta opp kampen.

\begin{remark}
    \textbf{ADVOKAT PÅ SPEEDDIAL} Visste du at som Tekna-medlem så kan du ringe 22 94 75 00 så har du en advokat på speeddial? Man kan også gjøre tilsvarende ved å sende mail til \url{juridisk@tekna.no}. Dette anbefaler jeg sterkt da de går gjennom arbeidskontrakter og bistår med å forhandle lønn. Du er derfor aldri alene!
\end{remark}

\begin{figure}[H]
    \centering
    \includegraphics[width=1\linewidth]{images/Sommerjobblønn.pdf}
    \caption{Anbefalt sommerjobblønn fra Tekna basert på klassetrinn}
\end{figure}



\section{Stipendiat}

Det finnes mange fordeler og ulemper med å bli en stipendiat (aka. ta en PhD). En spesiell ting med PhD i Norge er at det regnes som å være blant verdens beste mtp. lønn \cite{phd_salary_academiainsider}. I Norge ansees det som en fulltidsstilling i likhet med jobb i industrien. Derfor er det svært attraktivt for utenlandske studenter å ta PhD i Norge, men det er også den samme grunnen til hvorfor mange norske ikke velger å ta en PhD.

PhD-er regnes som offentlige stillinger og derfor er lønnen regulert av Staten. Et unntak er såkalte nærings-PhD-er hvor du er ansatt hos en bedrift og de betaler deg for å forske på noe konkret for dem, men i samarbeid med et universitet. Slike nærings-PhD-er er derfor <<best-of-both-worlds>>, men det er ikke så utbredt som vanlige PhD-stillinger. I Norge er PhD-stillinger på enten 3 eller 4 år hvis du i tillegg underviser. Plottet under viser hvordan PhD-lønnen har vært de siste årene. Vi ser at det er økende, men likevel svært mye lavere enn hos privat sektor. Eksempelvis så ligger 2023-lønnen over 100 000 kr lavere enn Teknas anbefaling. Derfor ansees PhD-stillinger ofte som en backup-løsning for noen, eller at det er for de som er spesielt interesserte.  

\begin{figure}[H]
    \centering
    \includegraphics[width=1\textwidth]{images/PhD-lønn.pdf}
    \caption{Minimum startlønn for stipendiater fra 2017 til 2023\cite{PHD}}
    \label{fig:PHD-lønn}
\end{figure}

Det er også verdt å merke at PhD-er defineres som en midlertidig fast stilling, aka. du har en kontrakt og mister jobben ved kontraktens slutt. Akademia er svært preget av slike midlertidige stillinger ettersom det skaper usikkerhet rundt ens jobbsituasjon. Post-doc er også en slik midlertidlig stilling, og er ofte bare en kontinuasjon etter PhD-en. Hvis du derimot skal bli i akademia, så må man vente på at en fast stilling blir ledig. Du har kanskje merket at folk i akademia ofte blir i akademia, noe som vil si at for at en ny stilling skal åpne seg så må enten en fast ansatt pensjonere seg eller så må instituttet plutselig få mer midler (ikke at det skjer så ofte heller) \cite{Tonnessen2023}. Derfor er realiteten at folk etter fullført PhD ofte går inn i industrien. Dette kan derfor i verstefall føre til at de får en lavere stilling enn det de hadde hatt om de gikk direkte til industri etter mastergraden. Men igjen, for noen er det viktig å ha PhD ved siden av navnet (det er jo ganske kult da). 


\section{Konsulent}

Konsulentbransjen i sin helhet har svært varierende lønningsnivåer, både fordi de driver med ulike typer konsulenttjenester og fordi det er ulike utdanninger som søker. \textbf{Ingeniør konsulenter} kompenserer ganske nært Tekna sin anbefaling, ifølge studenter jeg har snakket med. 

IT-konsulentene kan være litt mer gjerrige med lønnene sine og har en tendens til å gi ansatte en liten mengde lavere enn hva Tekna anbefaler. Dette er fordi mange bachelorgrader og folk med ikke-siving-grader søker som IT-konsulent og dermed blir lønningene lavere enn Teknas anbefaling. Et utdrag fra Abakus viser dette: \textit{Gjennomsnittslønnen ligger på rundt 615 000 kroner, som er 15 000 kroner lavere enn Teknas anbefalte begynnerlønn for 2023} \cite{abakus_utmatrikuleringsundersokelse}.  

Når det gjelder de resterende konsulentene, altså management, så er det ganske bredt hva de tjener. La oss starte med MBB, hvor de har sinnsyke lønninger ettersom overtid er innbakt i kontrakten deres \cite{dn_nyutdannede_konsulenter}. 

\begin{outline}
    Utdrag fra DN:
    \begin{itemize}
        \item Boston Consulting Group tilbyr en grunnlønn på rett under 790.000 kroner, etter det DN kjenner til.
        \item McKinsey har en grunnlønn på rundt 770.000 kroner, etter det DN kjenner til.
        \item Bain opplyser at grunnlønnen for nyutdannede ligger på omtrent samme nivå som de to konkurrentene.
    \end{itemize}
\end{outline}

Det skal da sies at hvis man jobber i McKinsey, så snitter man på 55 timer jobb i uka. Hvis man regner om 780K med 55 timer i uka til 40 timer i uka så nærmer man seg 570K. Derfor virker lønna mye, men i realiteten så er det dårligere timesbetalt enn Tekna sin anbefaling. 

Big4 har en tendens til å betale ganske dårlig ettersom de tar imot alle utdanningsbakgrunner. Gulroten i Big 4 er muligheten til å jobbe overtid, og det er her mulighetene ligger \cite{dn_nyutdannede_konsulenter_2024}.

\begin{outline}
    Etter 2023 tall så er lønningene på 570 000 kr. Resten er hentet fra DN.
    \begin{itemize}
        \item KPMG tilbød en startlønn på 560.000 kroner i 2022. Året før var startlønnen på 530.000 kroner. Det innebar altså en økning på 30.000 kroner fra 2021 til 2022. 
        \item EY tilbød i likhet med KPMG en startlønn på 560.000 kroner for nyutdannede i 2022. Dette bekreftes av selskapet.
        \item PwC hadde en startlønn på 560.000 kroner i 2022, opp 30.000 kroner fra 2021 da startlønnen var på 530.000 kroner. Etter oppjusteringen i år ble startlønnen for nyansatte rett ut av studiene oppjustert med 10.000 kroner til 570.000 kroner, opplyser selskapet.
        \item Deloitte hadde en startlønn på 550.000 kroner før lønnsjusteringen i begynnelsen av sommeren 2023. Dette var altså 10.000 kroner lavere enn de andre tre selskapene.
    \end{itemize}
\end{outline}

Den største gulroten er muligens lønningene til partnerne. Partnere i slike selskaper har jobbet i mange år og når tittelen etter å ha blitt medeier. Dermed tar de deler av risikoen til selskapet, men får heftige lønninger. Eksempelvis så snitter en EY-partner på 6.5 mill. \cite{e24_partner_big4}, mens hos McKinsey ligger det på 11.1 mill. \cite{e24_consulting_partner_mbb}. Det er verdt å merke seg at tallene er fra 2022. 



\section{Industri (Diplomundersøkelser)}

Siden <<Industri>> er en såpass stor kategori er det derfor svært vanskelig å samle tall herfra. Derfor ville jeg anbefalt å kikke på ulike diplomundersøkelser og snakke med bedrifter på stands osv. Her er det en lenke til en rekke diplomundersøkelser:

\begin{itemize}
    \item HC \cite{hc_ntnu_diplom_2022}
    \item Bindeleddet \cite{bindeleddet_documents_archive}
    \item EMIL-link \cite{emil_diplom_2023}
    \item Abakus \cite{abakus_utmatrikuleringsundersokelse}
\end{itemize}

Hvis man vil ha enda mer solide tall vil jeg virkelig anbefale Tekna sin lønnstatistikk, da de er flinkere til å samle data enn det linjerforeninger er. 

\begin{figure}[H]
    \centering
    \includegraphics[width=0.8\linewidth]{images/Tekna-lønnstatistikk.png}
    \caption{Skjermbilde av Teknas lønnstatistikk}
\end{figure}

Hvis du derimot fortsatt er usikker på hvor godt en siving tjener i forhold til resten av befolkningen har jeg laget følgende plot:

\begin{figure}[H]
    \centering
    \includegraphics[width=1\linewidth]{images/Tekna-SSB-Lønn.pdf}
    \caption{Utvikling av Teknas begynnerlønnabefaling mot Norges gjennomsnittlønn}
\end{figure}

(verdt å merke seg at den reelle nybegynnerlønnen som regel ligger under anbefalingen, men de er ganske nærme og avviker med kanskje 10-20K)

