\chapter*{Forord og disclaimer}

{
\usefont{T1}{ptm}{m}{it}

Kjære student,

Takk for at du tar deg tid til å lese dette dokumentet, som jeg har lagt mye arbeid i. Jeg har skrevet det fordi jeg har sett et behov for å samle alle karrieretips og diverse i ett felles dokument. Det er synd at mange verdifulle tips og triks går tapt når folk uteksamineres. Hvis vi kan ha "kok" for øvinger, hvorfor kan vi ikke ha det samme for jobbsøking?

Dette dokumentet er en samling av tips jeg har oppdaget i løpet av mine år på Gløshaugen, samt erfaringer fra andre. De fleste tipsene gjelder også for andre studier, men hovedsakelig vil de være fra en MTKJ-ers perspektiv, ettersom det er studiet jeg følger. Det er verdt å nevne at jeg spesialiserer meg i kjemisk prosessteknologi, så noen tips kan være påvirket av det.

Det er også viktig å inkludere en disclaimer, da jeg på ingen måte er allvitende, og dette dokumentet sannsynligvis inneholder noen feil. Jeg tar ikke ansvar for eventuelle feil eller konsekvenser som måtte oppstå, så vær kritisk til informasjonen og bruk den med forsiktighet.

Til tross for mulige feil, er formålet med dette dokumentet å videreføre mine tips, slik at du kan ta bedre karrierevalg.

Sist, men ikke minst, en stor takk til Celine Hansen og alle andre bidragsytere som har hjulpet meg med å forme denne guiden.

Med vennlig hilsen,

\Large Jun Xing Li
}

\vspace{10cm}

Kildekoden til denne filen, samt Python-kode og CV-mal finner du på GitHub-en min.

\url{https://github.com/Xingern/karriereguide}



\newpage
\chapter*{TL;DR}

Denne siden er for deg som ikke orker å lese hele dette dokumentet som jeg har brukt flere måneder på og ofret mange netter for å ferdigstille, samtidig som jeg skrev spesialiserings- og masteroppgave (ja, jeg prøver å gaslighte deg). Likevel har jeg laget en liste over de viktigste delene du bør få med deg hvis du virkelig har sååå dårlig tid.


\section*{1. og 2.klasse}

I de første årene tenker man ofte at valg om fremtidig jobb er langt unna og ikke noe man trenger å bekymre seg for ennå. Men hvis du har ambisjoner, kan du allerede nå ta grep som vil lønne seg senere. Det beste du kan gjøre er å ta sommerjobber som ferievikar, for eksempel som prosessoperatør hos Hydro Årdal. Du kan også søke på studass.-stillinger for fag du har hatt, eller lignende fag fra bachelor-studier. Det gir deg litt ekstra cash i lommen og CV-snacks som gir deg et fortrinn når du senere skal konkurrere om internships med dine jevnaldrende. For de spesielt interesserte finnes det også en rekke sommerkurs og mentorprogrammer du kan melde deg på tidlig i studieløpet.

\section*{3. og 4.klasse}

I løpet av disse årene begynner du kanskje å stresse over internships. Det er viktig å fokusere på CV-en og søknadsteksten, da disse dokumenterer det meste du har gjort. Husk at du har oppnådd mye mer enn du tror. Å ha verv er som å jobbe i en mindre organisasjon, og det handler bare om hvordan du fremstiller det. For de med litt lavere snitt, er sommerjobber en numbers game – det handler om å sende ut flest mulig søknader, og ett tilbud er nok for å lykkes. Bruk også tid på å undersøke hva ulike bedrifter gjør og hva som skiller dem. Det er litt pinlig å møte opp på intervju med feil utgangspunkt. Ulike bransjer har forskjellige rekrutteringsprosesser, så gjør litt research for å unngå feil. Når du leter etter sommerjobber, finner du ofte også graduate-jobber. Disse bør du lagre slik at du vet når jobber legges ut. I jobbsøking er informasjon alt. Det er nå du blir attraktiv for arbeidsgivere og da anbefaler jeg deg virkelig å dra på BEDPRES! Gratis mat og drikke mot litt reklame fra bedriftene er en god deal. Jo flere erfaringer du får når du søker internships, jo bedre forutsetninger har du når du skal søke fulltidsjobber i 5. klasse, så gjør forarbeidet tidlig.

\section*{5.klasse}

Dette er det viktigste året, hvor alle kjemper om fastjobb. Allerede fra da du valgte spesialisering har du lukket noen dører, så nå gjelder det å finne ut hva dine sjanser er og hvor du bør fokusere mest. Snakk med kullet over deg hvis du kjenner noen, eller delta aktivt på bedpres/karrieredager. Vær proaktiv og utnytt alle muligheter til å bygge nettverk og få innsikt i arbeidsmarkedet.